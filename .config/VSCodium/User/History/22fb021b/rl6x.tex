\documentclass{article}
\usepackage{graphicx} % Required for inserting images
\usepackage{listings}
\usepackage{amsmath}
\usepackage{geometry}
\geometry{
    a4paper,
    total={170mm,257mm},
    left=20mm,
    top=20mm,
}

\setlength{\parindent}{0pt}
\setlength{\parskip}{\baselineskip}


\usepackage{fontspec}
\setmainfont[Ligatures=TeX]{Verdana}

\title{Limiti dell'Algoritmo EIG (Compito 7.2)}
\author{Edoardo Vassallo S4965918}
\date{4 Giugno 2023}

\begin{document}

\maketitle

In questo foglio, mostreremo come l'algoritmo Exponential Information Gathering (EIG) è incapace di neutralizzare gli effetti del generale traditore nel problema dell'Accordo Bizantino, nel caso in cui sono presenti due generali leali ed un traditore. \\
Questo avviene sia nel caso in cui i generali siano inizialmente d'accordo, sia nel caso in cui non lo siano.

\section{Generali Concordi}
In questa sezione analizzeremo il caso in cui i generali sono concordi.\\

Nei nostri esempi, useremo le seguenti regole:
\begin{itemize}
    \item I generali saranno chiamati $A, B, C$.
    \item Il generale traditore sarà sempre $C$.
    \item Ogni generale può inviare il segnale $1$ o $0$.
    \item $C$ invierà nella prima comunicazione lo stesso segnale $x$ ad $A$ e $B$, allo scopo di non farsi scoprire. Nella seconda comunicazione dovrà inviare il bit corretto sul generale con cui sta comunicando, mentre potrà mentire, inviando i bit $y$ e $z$, sul generale rimanente.
    \item La decisione finale del generale $A$ sarà chiamata $\lambda$, quella di $B$ sarà $\mu$.
    \item In caso di pareggio nella valutazione dell'albero, viene scelto un valore di default concordato fra i generali leali, che chiamiamo $?$.
\end{itemize}   

Come esempio, prendiamo la comunicazione in cui i generali leali sono concordi sul bit $1$. Gli alberi creati, secondo l'algoritmo $EIG$, dai due nodi avranno forma:

\begin{center}
    \includegraphics[width=120mm]{EIG-A1B1.png}
\end{center}

In questo caso, basta che $C$ invii i seguenti messaggi:

\begin{align*}
    x = y &= 0\\
    z &= 1
\end{align*}

Gli alberi ottenuti ricalcolando i valori ad ogni nodo hanno forma:

\begin{center}
    \includegraphics[width=120mm]{EIG-A1B1-C001.png}
\end{center}

Osserviamo che tramite questi messaggi, l'esito della comunicazione dipende dal valore di $?$. Nel caso in cui $? = 0$, i generali ottengono valori discordi, nel caso in cui $? = 1$ essi concordano sul valore di default.\\

Nel caso concorde, è addirittura possibile fare in modo che i generali ottengano un valore opposto rispetto a quello su cui erano inizialmente d'accordo. 
Questo si può ad esempio ottenere nel caso in cui i rispettivi bit valgano:

\begin{align*}
    A&\to 1 \\
    B&\to 1 \\
    C&: \; x = y = z = 0\\
    ?& = 0
\end{align*}

In questo caso, gli alberi ricalcolati dai generali leali avrebbero forma:

\begin{center}
    \includegraphics[width=120mm]{EIG-A1B1-C000.png}
\end{center}

Questo vìola ovviamente il vincolo di validità del protocollo.

\section{Generali Discordi}

Nel nostro esempio, $A$ invierà il bit $1$, $B$ il bit $0$. Gli alberi costruiti avranno forma:

\begin{center}
    \includegraphics[width=120mm]{EIG-A1B0.png}
\end{center}

Il sabotatore può nuovamente riuscire nel suo intento inviando i seguenti messaggi:

\begin{align*}
    y = z &= 0\\
    x &= 1
\end{align*}

In questo caso, gli alberi ricalcolati avranno forma:

\begin{center}
    \includegraphics[width=120mm]{EIG-A1B0-C100.png}
\end{center}

Si può notare che il valore di $y$ è in realtà irrilevante.\\
\\
Similmente a prima, il risultato è dipendente dal valore di default $?$. Nel caso in cui $?=0$, i generali ottengono valori discordi, nel caso in cui $? = 1$ essi concordano sul valore di default \\

\section{Commenti}

Abbiamo dimostrato che il sabotatore è in grado di inviare messaggi capaci di sabotare la comunicazione sulla base del valore di default adottato dai generali leali.
La dipendenza della correttezza della comunicazione dal valore di default va direttamente contro uno dei due vincoli sulla base dei quali il protocollo è pensato, ovvero il vincolo di validità\\

Questo testimonia quindi l'incapacità dell'algoritmo nel ritrovare un accordo nel caso in cui i generali leali $l$ ed i generali traditori $f$ siano in rapporto:

\begin{align*}
    l < 2f + 1
\end{align*}

\end{document}
